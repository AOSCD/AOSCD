% Options for packages loaded elsewhere
\PassOptionsToPackage{unicode}{hyperref}
\PassOptionsToPackage{hyphens}{url}
%
\documentclass[
]{book}
\usepackage{lmodern}
\usepackage{amssymb,amsmath}
\usepackage{ifxetex,ifluatex}
\ifnum 0\ifxetex 1\fi\ifluatex 1\fi=0 % if pdftex
  \usepackage[T1]{fontenc}
  \usepackage[utf8]{inputenc}
  \usepackage{textcomp} % provide euro and other symbols
\else % if luatex or xetex
  \usepackage{unicode-math}
  \defaultfontfeatures{Scale=MatchLowercase}
  \defaultfontfeatures[\rmfamily]{Ligatures=TeX,Scale=1}
\fi
% Use upquote if available, for straight quotes in verbatim environments
\IfFileExists{upquote.sty}{\usepackage{upquote}}{}
\IfFileExists{microtype.sty}{% use microtype if available
  \usepackage[]{microtype}
  \UseMicrotypeSet[protrusion]{basicmath} % disable protrusion for tt fonts
}{}
\makeatletter
\@ifundefined{KOMAClassName}{% if non-KOMA class
  \IfFileExists{parskip.sty}{%
    \usepackage{parskip}
  }{% else
    \setlength{\parindent}{0pt}
    \setlength{\parskip}{6pt plus 2pt minus 1pt}}
}{% if KOMA class
  \KOMAoptions{parskip=half}}
\makeatother
\usepackage{xcolor}
\IfFileExists{xurl.sty}{\usepackage{xurl}}{} % add URL line breaks if available
\IfFileExists{bookmark.sty}{\usepackage{bookmark}}{\usepackage{hyperref}}
\hypersetup{
  pdftitle={Alternativas Open Source para ciencia de datos},
  pdfauthor={Daniel E. de la Rosa},
  hidelinks,
  pdfcreator={LaTeX via pandoc}}
\urlstyle{same} % disable monospaced font for URLs
\usepackage{longtable,booktabs}
% Correct order of tables after \paragraph or \subparagraph
\usepackage{etoolbox}
\makeatletter
\patchcmd\longtable{\par}{\if@noskipsec\mbox{}\fi\par}{}{}
\makeatother
% Allow footnotes in longtable head/foot
\IfFileExists{footnotehyper.sty}{\usepackage{footnotehyper}}{\usepackage{footnote}}
\makesavenoteenv{longtable}
\usepackage{graphicx}
\makeatletter
\def\maxwidth{\ifdim\Gin@nat@width>\linewidth\linewidth\else\Gin@nat@width\fi}
\def\maxheight{\ifdim\Gin@nat@height>\textheight\textheight\else\Gin@nat@height\fi}
\makeatother
% Scale images if necessary, so that they will not overflow the page
% margins by default, and it is still possible to overwrite the defaults
% using explicit options in \includegraphics[width, height, ...]{}
\setkeys{Gin}{width=\maxwidth,height=\maxheight,keepaspectratio}
% Set default figure placement to htbp
\makeatletter
\def\fps@figure{htbp}
\makeatother
\setlength{\emergencystretch}{3em} % prevent overfull lines
\providecommand{\tightlist}{%
  \setlength{\itemsep}{0pt}\setlength{\parskip}{0pt}}
\setcounter{secnumdepth}{5}
\usepackage{booktabs}
\usepackage[]{natbib}
\bibliographystyle{apalike}

\title{Alternativas Open Source para ciencia de datos}
\author{Daniel E. de la Rosa}
\date{2020-11-07}

\begin{document}
\maketitle

{
\setcounter{tocdepth}{1}
\tableofcontents
}
\hypertarget{presentaciuxf3n}{%
\chapter*{Presentación}\label{presentaciuxf3n}}
\addcontentsline{toc}{chapter}{Presentación}

Este libro tiene como objetivo principal el de evaluar el uso en primer lugar de R como alternativa a softwares privativos para la realización de ciencia de datos. Este interés responde a múltiples motivaciones, detro de las que podemos mencionar:

\begin{itemize}
\item
  \textbf{Interoperabilidad}: cuanto menor sea el número de herramientas y/o lenguajes necesarios para resolver un problema, más eficiente y efectiva resulta la solución. Aún si múltiples lenguales son necesarios, la posibilidad de poder utilizar todos o la gran mayoría de ellos en un mismo documento tiene una gran importancia.
\item
  \textbf{Costo}: el costo de las licencias de algunos de los softwares aquí cubiertos es quizás una las motiviaciones más importantes para la búsqueda de alternativas open source.
\item
  \textbf{Curva de aprendizaje}: como científicos de datos estamos llamados a manejar con cierta destreza los lenguajes de programación R y Python. No así el apredizaje de un nuevo lenguaje de programación o manejo de alguna herramienta tiene asociado un costo en tiempo y esfuerzo, que se convierte en una mativación más para aprender a resolver los problemas más comunes con las herramientas habituales.
\end{itemize}

Este documento, en principio, es por y para los estudiantes de la Maestría de Ciencias de Datos del INTEC y no pretende cubrir exhaustivamente ningun software o lenguaje de programación, sino más bien ofrecer soluciones alternativas, primordialmente en R, a los problemas y ejercicios de la maestría desarrollados en softwares propietarios.

La estructura del documento irá cambiando a lo largo del tiempo para ajustarse a las necesidades, siempre con el interés de agrupar los conceptos acorde a una temática en particular en un mismo capítulo.

\hypertarget{sintaxis}{%
\chapter{Sintaxis}\label{sintaxis}}

Esta capítulo reune los aspectos de sintaxis de todos los lenguajes utilizados a lo largo del libro. Elementos como la inserción de comentarios, la creación de objetos básicos, entre otros que sean prácticamente común a cualquier implementación del lenguaje.

\hypertarget{comentarios}{%
\section{Comentarios}\label{comentarios}}

Los comentarios en los lenguajes de programación representan texto legible que sirve para orientar al programador pero que en principio son obviados por el lenguaje. Por lo general son utilizados para describir el código a su alrededor.

\hypertarget{r}{%
\subsection{R}\label{r}}

En un script básico de R, un comentario puede ser insertado utilizando el símbolo de número \textbf{\texttt{\#}}.

\hypertarget{python}{%
\subsection{Python}\label{python}}

En Python los comentarios también se insertan utilizando el signo de número \textbf{\texttt{\#}}.

\hypertarget{matlab}}.

\hypertarget{octave}}.

  \bibliography{book.bib,packages.bib}

\end{document}
